\documentclass[cjk]{beamer}
% \documentclass[handout]{beamer}

\usepackage{beamerthemesplit}
\usepackage{xcolor}
\usepackage{tikz}
\usetikzlibrary{decorations.pathreplacing,positioning,arrows}
\usepackage{graphicx,wrapfig}

\defbeamertemplate*{footline}{}
{
  \leavevmode%
  \hbox{%
  \begin{beamercolorbox}[wd=.3\paperwidth,ht=2.25ex,dp=1ex,center]{author in head/foot}%
    \usebeamerfont{author in head/foot}\insertshortauthor  \end{beamercolorbox}%
  \begin{beamercolorbox}[wd=.6\paperwidth,ht=2.25ex,dp=1ex,center]{title in head/foot}%
    \usebeamerfont{title in head/foot}\insertshorttitle
  \end{beamercolorbox}%
  \begin{beamercolorbox}[wd=.1\paperwidth,ht=2.25ex,dp=1ex,right]{date in head/foot}%
    \insertframenumber{} / \inserttotalframenumber\hspace*{2ex} 
  \end{beamercolorbox}}%
  \vskip0pt%
}

\beamertemplateshadingbackground{red!10}{blue!10}

\addtobeamertemplate{frametitle}{}{%
\begin{tikzpicture}[remember picture,overlay] 
\node[anchor=north east,xshift=2pt,yshift=0pt] at (current page.north east)
{\includegraphics[height=0.35cm]{michiganrosslogo.pdf}}; 
\end{tikzpicture}}

\newcommand{\stick}{
\draw[thick] (0,0) circle [radius=10pt];
\draw[thick] (0,-0.35) -- (0,-1.5); 
\draw[thick] (0,-1.5) -- (-0.5,-2.5);
\draw[thick] (0,-1.5) -- (0.5,-2.5);
\draw[thick] (0,-0.8) -- (0.6,-0.6);
\draw[thick] (0,-0.8) -- (-0.6,-0.6);
}

\newcommand{\drawdemandcurve}[1]
{\draw[->] (0,0) -- (10,0);
\draw[->] (0,0) -- (0,5);
\draw (10,0.5) node{Quantity};
\draw (0,5.5) node{Price};
\draw[thick, color=#1] plot [smooth] coordinates { (0,4.8) (4,3) (7,2) (10,1.5) };
}

\renewcommand{\note}[1]{\emph{\small Note: #1}}

\begin{document}

\frame{

\frametitle{Main Point of the Paper}

\begin{itemize}
\item What is the main point of the paper?

\bigskip
\note{It's often harder than one may imagine to try and distill the main point of a paper. Another way to phrase this is ``What is the most important thing I learned from this paper?" The answer to the latter question is sometimes different from what the authors say the main point of the paper is.}

\bigskip
\begin{enumerate}[(a)]
\item Why is this an interesting or important idea? 

\bigskip
\note{The ``why do we care" question. Why are {\bf you} interested in this paper?}


\bigskip
\item What has previous work done in this area? What is the new contribution of this paper?

\bigskip
\note{May spill over to another slide by now.}
\end{enumerate}

\end{itemize}

}

\frame{

\frametitle{Setting}

\begin{itemize}
\item If theory paper, what's the main intuition? 

\bigskip
\note{Can you identify the main friction that leads to the results? Can you briefly explain the intuition? Is it a good model for the phenomenon being studied (does it capture what you think are the first-order effects)?}

\bigskip
\item If empirical paper, what is the setting? 

\bigskip
\note{(1) Reduced form empirical: Why is this a good setting for the test? What is the main identification assumption? Is the exclusion restriction likely to hold? External validity? \\

(2) Structural papers, of course, have some of both theory and empirics, so you'll have to do a little more work here. What are the main structural parameters being estimated? Do they relate to preferences/costs/technology/something else? Why is a structural model useful in this context?}

\end{itemize}

}


\frame{

\frametitle{Method}

\begin{itemize}

\item What is unique/clever about the method or setting? 

\bigskip
\emph{Only use this slide if there is something you learned from this paper that carries over broadly in terms of a method you can use later in future work. Otherwise, skip and move on.}

\end{itemize}

}


\frame{

\frametitle{Main Result}

\begin{itemize}
\item If theory paper, what is the main proposition?


\bigskip
\item If empirical, what is the main table? Show the key coefficients. 

\bigskip
\note{\begin{enumerate}
\item At most two tables/propositions here.

\item Do the main propositions/tables tell us the story that was promised? Or do they not quite get there? 

\end{enumerate}}

\end{itemize}

}

\frame{

\frametitle{Discussion/Critique}

\begin{itemize}
\item What is good about what the authors did?

\bigskip
\note{E.g., Answered a really important question / Clean, elegant model / Well-designed structural model / Found a good exogenous shock.}

\bigskip
\item What can be improved on?

\bigskip
\note{E.g., Ask a more interesting question / Endogenize some assumption / Use the data in some different way.}

\end{itemize}

}

\frame{

\frametitle{Open Questions in this Area}

\begin{itemize}
\item What is worth exploring going forward in this area? Does the paper open the door to further research?

\bigskip
\note{If you could design a ``dream result" in this area, what would that be?}

\end{itemize}

}

\frame{

\frametitle{Bonus Slide}

\begin{itemize}
\item Not for reading group use, but I just noticed that my template file for slides has some basic tikz objects pre-defined. 

\bigskip
\begin{tikzpicture}[thick,scale=0.9]

\drawdemandcurve{red}

\begin{scope}[shift={(3,3)},scale=0.5]
{\color{blue} \stick}
\end{scope}

\begin{scope}[shift={(6,4.5)},scale=0.75]
{\color{green!60!black} \stick}
\end{scope}

\end{tikzpicture}

\end{itemize}

}


\end{document}

